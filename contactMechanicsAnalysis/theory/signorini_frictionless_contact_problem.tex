\documentclass[10pt,a4paper]{article}
\usepackage[utf8]{inputenc}
\usepackage{amsmath}
\usepackage{amsfonts}
\usepackage{amssymb}
\usepackage{import}
\usepackage{graphicx}
\usepackage{color}

\setlength{\parindent}{0cm}

% Define own macros
\newcommand{\myfrac}[2]{%
    \ifinner#1/#2%
    \else\frac{#1}{#2}%
    \fi%
}

\title{Signorini frictionless contact problem}

\author{Dr.-Ing. Andreas Apostolatos}

\date{\today}

\begin{document}

\maketitle

\section{Theory}\label{sec:variational_formulation}

\begin{figure}[!h]
	\centering
	\footnotesize
    \def\svgwidth{0.5\textwidth}\import{./figures/}{problem_placement.pdf_tex}
	\caption{Theory: Signorini frictionless contact problem with boundary conditions.}
	\label{im:signorini_contact_problem}
\end{figure}

Given is a deformable body which is geometrically described by $\Omega \subset \mathbb{R}^d$ where $d = 2,3$ stands for the problem's dimensions. Its deformation due to the applied boundary conditions can be uniquely defined by a displacement field $\mathbf{u} = u_i \, \mathbf{e}_i$, $\mathbf{e}_i$ being the Cartesian basis, which maps each each material point of the reference configuration $\mathbf{X} \in \Omega$ to a material point in the current configuration $\mathbf{x} \in \Omega_t$, that is, $\mathbf{x} = \mathbf{X} + \mathbf{u}$.\\

The strain is described by means of the \textit{Green-Lagrange} (GL) strain second order tensor $\mathbf{E} \in \mathfrak{S}^2$ given by,

\begin{equation}
	\mathbf{E} = \frac{1}{2} \left( \boldsymbol{\nabla} \mathbf{u} + \left( \boldsymbol{\nabla} \mathbf{u} \right)^{\text{t}} +  \boldsymbol{\nabla} \mathbf{u} \, \vdots \, \left( \boldsymbol{\nabla} \mathbf{u} \right)^{\text{t}} \right) \;, \label{eq:GL_strain}
\end{equation}

where the underlying symbols are understood as follows,

\begin{subequations}
	\begin{alignat}{1}
		\boldsymbol{\nabla} \mathbf{u} &= \sum_{i = 1}^d \sum_{j = 1}^d \frac{\partial u_i}{\partial X_j} \, \mathbf{e}_i \otimes \mathbf{e}_j \;, \label{eq:nabla_u} \\
		\left( \boldsymbol{\nabla} \mathbf{u} \right)^{\text{t}} &= \sum_{i = 1}^d \sum_{j = 1}^d \frac{\partial u_j}{\partial X_i} \, \mathbf{e}_i \otimes \mathbf{e}_j \;, \label{eq:nabla_u_t} \\
		\boldsymbol{\nabla} \mathbf{u} \, \vdots \left( \boldsymbol{\nabla} \mathbf{u} \right)^{\text{t}} &= \sum_{i = 1}^d \sum_{j = 1}^d \frac{\partial u_i}{\partial X_j} \frac{\partial u_j}{\partial X_i} \;. \label{eq:nabla_u_times_u_t}
	\end{alignat}
\end{subequations}

The stress state of the problem is described by means of the 2nd \textit{Piola-Kirchhoff} (PK2) stress second order ternsor $\mathbf{S} \in \mathfrak{S}^2$ which is defined,

\end{document}