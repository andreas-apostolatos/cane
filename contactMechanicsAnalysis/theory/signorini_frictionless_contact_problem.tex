\documentclass[10pt,a4paper]{article}
\usepackage[utf8]{inputenc}
\usepackage{amsmath}
\usepackage{amsfonts}
\usepackage{amssymb}
\usepackage{import}
\usepackage{graphicx}
\usepackage{color}

\setlength{\parindent}{0cm}

% Define own macros
\newcommand{\myfrac}[2]{%
    \ifinner#1/#2%
    \else\frac{#1}{#2}%
    \fi%
}

\title{Signorini frictionless contact problem}

\author{Dr.-Ing. Andreas Apostolatos}

\date{\today}

\begin{document}

\maketitle

\section{Theory}\label{sec:variational_formulation}

Given is a deformable body which is geometrically described by $\Omega \subset \mathbb{R}^d$ where $d = 2,3$ stands for the problem's dimensions. Its deformation due to the applied boundary conditions can be uniquely defined by a displacement field $\mathbf{u} = u_i \, \mathbf{e}_i$, $\mathbf{e}_i$ being the Cartesian basis, which maps each each material point of the reference configuration $\mathbf{X} \in \Omega$ to a material point in the current configuration $\mathbf{x} \in \Omega_t$, that is, $\mathbf{x} = \mathbf{X} + \mathbf{u}$.\\

The strain is described by means of the \textit{Green-Lagrange} (GL) strain second order tensor $\boldsymbol{\mathcal{E}} \in \mathfrak{S}^2$ given by,

\begin{equation}
	\boldsymbol{\mathcal{E}} = \frac{1}{2} \left( \boldsymbol{\nabla} \mathbf{u} + \left( \boldsymbol{\nabla} \mathbf{u} \right)^{\text{t}} +  \boldsymbol{\nabla} \mathbf{u} \cdot \left( \boldsymbol{\nabla} \mathbf{u} \right)^{\text{t}} \right) \;, \label{eq:GL_strain}
\end{equation}

where the underlying symbols are understood as follows,

\begin{subequations}
	\begin{alignat}{1}
		\boldsymbol{\nabla} \mathbf{u} &= \sum_{i = 1}^d \sum_{j = 1}^d \frac{\partial u_i}{\partial X_j} \, \mathbf{e}_i \otimes \mathbf{e}_j \;, \label{eq:nabla_u} \\
		\left( \boldsymbol{\nabla} \mathbf{u} \right)^{\text{t}} &= \sum_{k = 1}^d \sum_{l = 1}^d \frac{\partial u_l}{\partial X_k} \, \mathbf{e}_k \otimes \mathbf{e}_l \;, \label{eq:nabla_u_t} \\
		\boldsymbol{\nabla} \mathbf{u} \cdot \left( \boldsymbol{\nabla} \mathbf{u} \right)^{\text{t}} &= \sum_{i = 1}^d \sum_{j = 1}^d \sum_{k = 1}^d \frac{\partial u_i}{\partial X_j} \frac{\partial u_j}{\partial X_k} \, \mathbf{e}_i \otimes \mathbf{e}_k \;. \label{eq:nabla_u_times_u_t}
	\end{alignat}
\end{subequations}

Assumed is also a Saint-Venant material model which is governed by the material fourth order tensor $\boldsymbol{\mathcal{C}} \in \mathfrak{S}^2$

\begin{equation}
	\boldsymbol{\mathcal{C}} = \frac{E}{2 (1 + \nu)} \left( \delta_{ik} \delta_{jl} + \delta_{il} \delta_{jk} + \frac{2 \nu}{1 - 2 \nu} \delta_{ij} \delta_{kl} \right) \, \mathbf{e}_i \otimes \mathbf{e}_j \otimes \mathbf{e}_k \otimes \mathbf{e}_l \;, \label{eq:material_tensor}
\end{equation}

where $E$ and $\nu$ stand for the Young's modulus and Poisson ratio of the material, respectively. The stress state of the problem is described by means of the 2nd \textit{Piola-Kirchhoff} (PK2) stress second order ternsor $\boldsymbol{\mathcal{S}} \in \mathfrak{S}^2$ which is defined as and given by the linear elastic isotropic law as,

\begin{equation}
	\boldsymbol{\mathcal{S}} = \boldsymbol{\mathcal{C}}  : \boldsymbol{\mathcal{E}} \:.
\end{equation}

\begin{figure}[!t]
	\centering
	\footnotesize
    \def\svgwidth{0.5\textwidth}\import{./figures/}{problem_placement.pdf_tex}
	\caption{Theory: Signorini frictionless contact problem with boundary conditions.}
	\label{im:signorini_contact_problem}
\end{figure}

Given the Dirichlet boundary conditions along $\Gamma_{\text{d}}$ where the displacement is prescribed to a value $\mathbf{g}$, the Neumann boundary conditions along $\Gamma_{\text{n}}$ where external traction $\bar{\mathbf{t}}$ is applied and applied body forces $\mathbf{b}$, see Fig.~\ref{im:signorini_contact_problem}, the strong form of the elasticity problem writes,

\begin{subequations}
	\begin{alignat}{2}
		\boldsymbol{\nabla} \cdot \boldsymbol{\mathcal{S}} + \mathbf{b} &= \mathbf{0} \quad &&\text{in } \Omega \;, \label{eq:strong_equilibrium} \\
		\mathbf{u} &= \mathbf{g} &&\text{on } \Gamma_{\text{d}} \;, \label{eq:prescribed_displacement} \\
		\boldsymbol{\mathcal{S}} \cdot \mathbf{n} &= \bar{\mathbf{t}} &&\text{on } \Gamma_{\text{n}} \;,
	\end{alignat}
	\label{eq:strong_formulation}
\end{subequations}

where $\mathbf{n}$ stands for the unit outward normal vector to Neumann boundary $\Gamma_{\text{n}}$.\\

Assumed is also that there is a boundary $\Gamma_{\text{c}} = \Gamma_{\text{c}} (\mathbf{u})$ along which body $\Omega$ is expected to come into contact with a rigid body $\mathfrak{F}$. The so-called gap function is then defined along $\Gamma_{\text{c}}$, namely $\mathbf{g}:\Gamma_{\text{c}} \rightarrow \mathbb{R}$ and measures the normal distance between the deformable and the rigid body $\Omega$ along $\Gamma_{\text{c}}$ and $\mathfrak{F}$, respectively. In this way, the so-called complementarity conditions can be stated as follows,

\begin{subequations}
	\begin{alignat}{1}
		u_n - g_n &\le 0 \;, \label{eq:complementarity_cnd_gap} \\
		t_n &\ge 0 \;,  \label{eq:complementarity_cnd_traction} \\
		t_n\,(u_n - g_n) &= 0 \label{eq:complementarity_cnd_mixed}
	\end{alignat}
	\label{eq:complementarity_cnds}
\end{subequations}

where $t_n = \mathbf{n} \cdot \boldsymbol{\mathcal{S}} \cdot \mathbf{n}$, $u_n = \mathbf{u} \cdot \mathbf{n}$ and $g_n = \mathbf{g} \cdot \mathbf{n}$ stand for is the normal to the boundary $\Gamma_{\text{c}}$ traction vector, the normal to the boundary $\Gamma_{\text{c}}$ component of the displacement field and the gap function with respect to the outward unit normal vector $\mathbf{n}$. Condition in Eq.~\eqref{eq:complementarity_cnd_gap} stands for the non-penetration condition whereas condition in Eq.~\eqref{eq:complementarity_cnd_traction} states that the stresses along the contact boundary need to be compressive.\\

The weak formulation of the problem in Eq.~\eqref{eq:strong_formulation} subject to the complementarity conditions in Eq.~\eqref{eq:complementarity_cnds} can be written by means of the Langrange Multipliers method: Find $\mathbf{u} \in \boldsymbol{\mathcal{H}}^1 (\Omega)$ with $\mathbf{u} = \mathbf{g}$ on $\Gamma_{\text{d}}$ and $\lambda \in \mathcal{L}^2(\Gamma_{\text{c}})$ with $\lambda \ge 0$ such that,

\begin{subequations}
	\begin{alignat}{1}
		\int_{\Omega} \delta \boldsymbol{\mathcal{E}} : \boldsymbol{\mathcal{S}} \: \text{d} \Omega + \int_{\Gamma_{\text{c}}} \delta u_n \, \lambda \: \text{d} \Gamma &= \int_{\Omega} \delta \mathbf{u} \cdot \mathbf{b} \: \text{d} \Omega + \int_{\Gamma_{\text{n}}} \bar{\mathbf{t}} \cdot \mathbf{u} \: \text{d} \Gamma \label{eq:variational_formulation_delta_u} \\
		\int_{\Gamma_{\text{c}}} \delta \lambda \, u_n \: \text{d} \Gamma &\le \int_{\Gamma_{\text{c}}} \delta \lambda \, g_n \: \text{d} \Gamma \;, \label{eq:variational_formulation_delta_lambda}
	\end{alignat}
\end{subequations}

for all $\delta \mathbf{u} \in \boldsymbol{\mathcal{H}}^1 (\Omega)$ and for all $\delta \lambda \in \mathcal{L}^2(\Gamma_{\text{c}})$ with $\delta \lambda \ge 0$. In fact, $\lambda$ in this case represents the normal traction along the contact boundary $\Gamma_{\text{c}}$ and therefore it must be positive, see Eq.~\eqref{eq:complementarity_cnd_traction}. Eq.~\eqref{eq:variational_formulation_delta_u} is nothing else but the internal virtual work in $\Omega$, whereas Eq.~\eqref{eq:variational_formulation_delta_lambda} accounts for the variation of complementarity condition in Eq.~\eqref{eq:complementarity_cnd_mixed} and renders the problem symmetric as it is also demonstrated in its discrete form.

\section{Discretization}

Assumed is that the displacement field is discretized with the standard linear \textit{Finite Element Method} (FEM) on triangles or quadrilaterals. Employing the Buvnon-Galerkin FEM then the unknown displacement field $\mathbf{u}$ and its variation $\delta \mathbf{u}$ are discretized using the same basis functions $\boldsymbol{\varphi}_i$, that is,

\begin{subequations}
	\begin{alignat}{1}
		\mathbf{u}_h &= \sum_{i = 1}^n \boldsymbol{\varphi}_i \hat{u}_i \;, \label{eq:discretization_u} \\
		\delta \mathbf{u}_h &= \sum_{i = 1}^n \boldsymbol{\varphi}_i \delta\hat{u}_i \;, \label{eq:discretization_delta_u}
	\end{alignat}
\end{subequations}

where the basis linear/bilinear functions $\boldsymbol{\varphi}_i$ are constructed as a dual product of the linear/bilinear shape functions $N_i$ at the element's parametric space and the Cartesian basis $\mathbf{e}_i$. The subscript $h$ indicates then the smallest element length size within the employed mesh.\\

Within the node-based contact method, the shape functions for the discretization of the Lagrange Multipliers fields $\lambda$ and $\delta \lambda$ are chosen the diract delta distributions supported on the finite element nodes $\mathbf{X}_i$, that is,

\begin{subequations}
	\begin{alignat}{1}
		\lambda &= \sum_{i = 1}^n \delta(\mathbf{X}_i) \boldsymbol{\varphi}_i \hat{u}_i \;, \label{eq:discretization_u} \\
		\delta \lambda_h &= \sum_{i = 1}^n \boldsymbol{\varphi}_i \delta\hat{u}_i \;, \label{eq:discretization_delta_u}
	\end{alignat}
\end{subequations}

\begin{figure}[!t]
	\centering
	\footnotesize
    \def\svgwidth{0.7\textwidth}\import{./figures/}{finite_element_mesh.pdf_tex}
	\caption{Discretization: Finite element mesh and degrees of freedom.}
	\label{im:finite_element_mesh}
\end{figure}

\end{document}